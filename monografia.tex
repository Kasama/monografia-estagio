\documentclass[12pt,times,a4paper,twoside]{icmc}

\usepackage[brazil]{babel}
\usepackage[utf8]{inputenc}
\usepackage[top=30mm,bottom=20mm,left=30mm,right=20mm,twoside]{geometry}
\usepackage[usenames,dvipsnames]{color}
\usepackage[nottoc]{tocbibind}
\usepackage{fancychap} % Remova se quiser tirar os detalhes do título do capítulo
\usepackage{indentfirst}
\usepackage{setspace}
\usepackage{graphicx}
\usepackage{epstopdf}
\usepackage{amssymb}
\usepackage{amsmath}
\usepackage{mathptmx} % Necessário para corrigir a fonte para Times
\usepackage{hyperref}
\usepackage{setspace}
\usepackage{algpseudocode}
\usepackage{pdfpages}
\usepackage{url}
\usepackage{listings}
\usepackage{natbib}
\lstset{numbers=left, %%% Para insercao de codigos e listagens
  stepnumber=1,
    firstnumber=1,
    numberstyle=\tiny,
    belowskip= 0.5cm,
    numbersep= 0.2cm,
    extendedchars=true,
    breaklines=true,
    frame=tb,
    basicstyle=\footnotesize,
    stringstyle=\ttfamily,
    showstringspaces=false
}

\usepackage{parskip}
\usepackage{longtable}
%\usepackage[toc]{glossaries}
%\makeglossaries%

\hypersetup{%
  colorlinks = true,
  citecolor = black,
  filecolor = black,
  linkcolor = black,
  urlcolor = black,
}

\usepackage[toc,acronyms]{glossaries}
\makeglossaries%

\input glossary.tex

\newcommand{\tituloMonografia}{Estágio em DevOps}

\newcommand{\nomeAluno}{Roberto Pommella Alegro}

\newcommand{\nomeSupervisor}{Rafael Pereira Girolineto}

\newcommand{\nomeEmpresa}{Casa e Café}

\renewcommand*{\lstlistlistingname}{Lista de Listagens}

\hyphenation{}

\begin{document}
\pagenumbering{roman}

\begin{titlepage}
\pagestyle{empty} % no headers in this environment

% title page
\begin{center}
\begin{minipage}[c]{12cm}
\begin{center}
\vspace{.35\textheight}
\hrulefill\\
    \vspace{.5cm} {\Large \textcolor{blue}{\tituloMonografia}}\\
    \vspace{1.3cm}
    \textbf{\nomeAluno}\\
      \vspace{.5cm}
      \hrulefill\\
        \vspace{5cm}
        \includegraphics[width=5cm]{img/logoICMC.png}
        \end{center}
        \end{minipage}
        \end{center}


        \cleardoublepage%


        \vspace*{3cm}
        \begin{center}
{\huge\sf \textcolor{blue}{\tituloMonografia}} \\
    \vspace*{2cm}
{\bf \nomeAluno} \\
    \vspace*{2cm}
    \emph{Supervisor:}  {\nomeSupervisor}\\
      \emph{Empresa:}  {\nomeEmpresa}
      \end{center}
      \vspace*{3cm}

      \begin{flushright}
      \begin{minipage}{10cm}
      Monografia de conclusão de curso apresentada ao
      Instituto de Ciências Matemáticas e de Computação da Universidade de São Paulo para obtenção do título de Bacharel em
      Ciências de Computação.
      \end{minipage}
      \end{flushright}

      \vspace*{2cm}
      \begin{center}
      \textbf{USP --- São Carlos \\ Novembro de 2017}
      \end{center}

      \cleardoublepage%

      \end{titlepage}

      \thispagestyle{plain}
      \setcounter{page}{1}

      \input dedicatoria.tex

      \input agradecimentos.tex

      %%%%%%%%%%%%%%%%%%%%%%%%%%

      \input resumo.tex

      \tableofcontents
      \listoffigures
      \listoftables
      \lstlistoflistings%
      \printglossary[title=Lista de Termos,toctitle=Termos e Abreviaturas]

      \onehalfspacing%

      \mainmatter%

      \renewcommand{\chaptermark}[1]{%

        \markboth{\chaptername\thechapter.\ #1}{}}  %

          \renewcommand{\sectionmark}[1]{%
            \markright{\thesection.\ #1}}

            %\input ...
            \input introducao.tex
            \input planejamento.tex
            \input desenvolvimento.tex
            \input conclusao.tex

            %Observação 1: É obrigatório que a monografia tenha uma lista de referências que deve estar contida neste item.
            %Observação 2: As referências devem estar em ordem alfabética pelo sobrenome do primeiro autor.
            %Observação 3: Todos os documentos referenciados nesse item (livros, artigos, relatórios técnicos, sites, entre outros) deverão ter sido citados no texto.
            %Observação 4: Todos as citações no decorrer do texto deverão ser listadas neste item.
            %Observação 5: É obrigatório que a lista de referências esteja de acordo com a norma NBR-6023/2002 para referências bibliográficas.

            \renewcommand{\bibname}{Referências}
            %\printbibliography[heading=bibintoc]
            \bibliographystyle{plain}
            \bibliography{referencias}

            % Se precisar de apêndice, use a seção abaixo
            %\appendix
            %\renewcommand\appendixname{Apêndice}
            %\renewcommand\chaptername{Apêndice}

            \end{document}
