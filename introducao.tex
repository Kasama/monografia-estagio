\chapter{Introdução}\label{chap:intro}
\pagenumbering{arabic}

\section{Sobre a Empresa}

O Casa e Café é uma agência de empregos domésticos online que prove uma plataforma na qual famílias podem buscar pelo profissional ideal para suas necessidades. Ao mesmo tempo que ajuda profissionais do ramo de trabalhos domésticos a encontrar um emprego. A plataforma suporta várias categorias de trabalhos domésticos, incluindo, mas não limitado a, empregados(as) domésticas, babás e cozinheiros(as).\\

A missão da empresa é prover paz para os clientes em casa através do acesso a profissionais de alta qualidade.\\

%% Comente sobre a empresa: o setor de atuação, o porte, a missão e outras informações que você julgue necessárias, como por exemplo plano de carreira e premiações.

%% Em toda a monografia, você pode fazer uso de  referencias bibliográficas para apresentar URLs~\cite{WAI}, livros~\cite{Dahl:1972}~\cite{Hopcroft:1969} ou artigos utilizados no decorrer do estágio, publicados em periódicos~\cite{Almorsy:2012} ou revistas~\cite{MichettiJavaMagazine2013}.

\section{Sobre o Processo Seletivo}

A vaga oferecida para estágio foi uma vaga de \gls{devops}, que é um termo utilizado para se referir ao profissional que trabalha com \gls{ops} e infraestrutura, o trabalho contempla desafios como manutenção e monitoramento de servidores e automação de atividades.\\

Para participar do processo de seleção, ocorreu uma indicação à vaga por um amigo, Gustavo Aguiar, que conheceu o \nomeSupervisor~em um evento. O processo seletivo incluiu uma entrevista técnica, um desafio prático envolvendo área de interesse e outras duas entrevistas com os outros sócios da empresa.\\

A preparação para as entrevistas incluiu estudos sobre a empresa e seu ramo de atividade, além de revisões em conceitos de infraestrutura e DevOps para responder a perguntas técnicas.\\

Para solucionar o desafio, foi necessário reforçar os conhecimentos prévios em Docker e sistemas Linux. O Rafael teve uma resposta positiva à entrevista e à solução do desafio, permitindo a continuidade do processo com entrevistas com os outros dois sócios e eventualmente numa proposta de estágio.\\

%% Comente sobre como foi o processo seletivo para admissão no estágio. Discuta como você se preparou para ele.

\section{Apresentação da monografia}

No Capítulo~\ref{chap:atividadesPlanejadas} dada uma breve introdução da motivação e problemas que influenciaram as tarefas. Em seguida um breve resumo das atividades previstas para o estágio, com um cronograma estimado.\\

No Capítulo~\ref{chap:atividadesRealizadas} são apresentados detalhes das atividades executadas no decorrer do estágio e seus impactos, bem como as ferramentas que possibilitaram a execução das tarefas e os problemas que não foram solucionados.\\

No Capítulo~\ref{chap:conclusao} é apresentada uma breve conclusão sobre os benefícios da experiência do estágio, algumas sugestões para a melhor preparação do curso de graduação para o mercado de trabalho e planos para o futuro.\\

%% Escreva um parágrafo que resume cada um dos demais capítulos da monografia, fazendo referência a cada um deles --- como no exemplo que segue. Indique também a existência de apêndices e anexos, se houver.

%% No Capítulo~\ref{chap:atividadesPlanejadas} é primeiramente sumarizado o planejamento das atividades previstas para todo o estágio, com o cronograma correspondente. A seguir, são apresentados  os treinamentos previstos.
