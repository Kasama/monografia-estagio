\chapter{Planejamento do Trabalho}\label{chap:atividadesPlanejadas}

O planejamento inicial do estágio incluiu a automação de geração de relatórios e execução de scripts de manutenção do sistema, como backups e migrações de dados. Melhorias na arquitetura e infraestrutura do sistema automatização de ambientes de teste e build de imagens \gls{Docker}. Monitoramento da saúde da plataforma e melhorias no processo interno para atingir \gls{CD}

%% Este capítulo não deve ultrapassar o total de 2 páginas.

\section{Atividades planejadas para o estágio}

% Discorra sobre quais atividades foram originariamente planejadas para o estágio, discutindo a quais problemas/objetivos estariam associados.

% Você pode fazer uso de tabelas, por exemplo para sumarizar o cronograma de atividades planejados. Quando você usar uma tabela (ou figura, ou quadro, ou listagem), é obrigatório fazer referência a ela quando seu conteúdo é explicado no texto, como por exemplo: ver Tabela~\ref{tab:tab1}.

Nas primeiras semanas do estágio, foram planejadas atividades de familiarização com a arquitetura existente e desenvolvimento de um novo~\gls{scheduler} de processos e scripts.\\

Nas semanas seguintes foram planejadas atividades de integração, nas quais foi usado o sistema \gls{Jenkins} em conjunto com \gls{EC2} do \gls{AWS} e também atividades de geração de relatórios e manutenção da plataforma, como backups da base de dados e o blog da empresa usando o \gls{scheduler} criado previamente.\\

Na tabela \ref{tab:tabPlan}, está ilustrado o cronograma de atividades planejadas em ordem cronológica.\\

\begin{table}[H]
\begin{center}
\caption[Atividades Planejadas]
{Atividades Planejadas}\label{tab:tabPlan}

\begin{tabular}{llp{7cm}} \hline

\hline
\textbf{Início}    & \textbf{Fim} &  \textbf{Descrição}                             \\
\hline
31/07/2017        & 10/08/2017      & Implementação do \gls{scheduler}\\
16/08/2017        & 31/08/2017      & Implementação de \gls{OAuth2} no \gls{backend}\\
04/09/2017        & 07/09/2017      & Implementação de \gls{OAuth2} no \gls{frontend}\\
% 11/09/2017        & 19/09/2017      & Ferias: Hackathon Hack The North\\
20/09/2017        & 28/09/2017      & Automatização da geração de relatórios\\
02/10/2017        & 05/10/2017      & \gls{POC} do \gls{Teste AB} da plataforma\\
09/10/2017        & 19/10/2017      & Implementação do \gls{Teste AB}\\
23/10/2017        & 25/10/2017      & Automatização do \gls{backup} do blog\\
23/10/2017        & 01/11/2017      & Automatização de testes e \gls{checkstyle} com \gls{Jenkins}\\
06/11/2017        & 09/11/2017      & Automatização do \gls{backup} da base de dados de produção\\
13/11/2017        & 16/11/2017      & Automatização do build de imagens do sistema\\
\hline

\hline
\end{tabular}
\end{center}
\end{table}

\section{Treinamentos planejados para o estágio}

Não houve um planejamento de treinamentos específicos, todo o treinamento foi dado conforme necessário para realizar as tarefas.\\

%
% Discorra sobre treinamentos previstos quando do início do estágio.
%
% Quando citar os treinamentos planejados, a duração de cada um pode ser apresentada em uma tabela, como ilustrado na Tabela~\ref{tab:tab2}.
%
% \begin{table}[!ht]
% \begin{center}
% \caption[Treinamentos Planejados]
% {Treinamentos Planejados}\label{tab:tab2}
%
% \begin{tabular}{llp{7cm}}
% \hline
%
% \hline
% \textbf{Início}    & \textbf{Fim} &  \textbf{Descrição}                             \\
% \hline
% 01/02/2014       & 15/02/2014                & Treinamento em Ruby on Rails \\
% 15/02/2014       & 28/02/2014                & Treinamento em SOA\\
%
% \hline
% \end{tabular}
% \end{center}
% \end{table}
