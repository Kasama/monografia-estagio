% abbreviations: coloque em ordem alfabética
\newacronym{AWS}{AWS}{Amazon Web Services}
\newacronym{BCC}{BCC}{Bacharelado em Ciências da Computação}
\newacronym{CD}{CD}{Continuous Deployment}
\newacronym{CEP}{CEP}{Código de Endereçamento Postal}
\newacronym{EC2}{EC2}{Elastic Compute Cloud}
\newacronym{ECS}{ECS}{\gls{EC2} Container Service}
\newacronym{ICMC}{ICMC}{Instituto de Ciências Matemáticas e de Computação}
\newacronym{LXC}{LXC}{Linux Containers}
\newacronym{SO}{SO}{Sistema Operacional}
\newacronym{SP}{SP}{São Paulo}
\newacronym{USP}{USP}{Universidade de São Paulo}
\newacronym{IO}{E/S}{Entrada e Saída}
\newacronym{TI}{TI}{Tecnologia da Informação}

\newglossaryentry{Docker}
{
  name=Docker,
  description={Docker é a empresa que está liderando o movimento do uso de \gslpl{container}. Docker permite a independencia entre aplicações e infraestrutura, facilitando a colaboração e inovação dos profissionais de desenvolvimento e de \gsl{ops}.},
  plural=Dockers
}

\newglossaryentry{container}
{
  name=container,
  plural=containers,
  description={\gls{LXC} é uma camada de virtualização no nível de \gls{SO} que permite a execução de múltiplos ambientes isolados no mesmo \gls{Kernel}
  }
}

\newglossaryentry{Kernel}
{
  name=Kernel,
  plural=Kernels,
  description={
    O Kernel é o programa de computador central de um \gls{SO}, que gerencia toda a comunicação entre os processos em alto nível e os dispositivos de \gls{IO}
  }
}

\newglossaryentry{ops}
{
  name=Ops,
  plural=Ops,
  description={
    Ops é o nome dado ao conjunto de processos e serviços que garantem o bom funcionamento da infraestrutura de um ambiente operacional em \gsl{TI}.
  }
}



