\chapter{Introdução}\label{chap:intro}
\pagenumbering{arabic}

O capítulo de Introdução deve ter no máximo o total de 7 páginas.

\section{Sobre a Empresa}

A Casa e Café é uma agencia de empregos domésticos online que prove uma plataforma na qual famílias podem buscar pelo profissional ideal para suas necessidades. Ao mesmo tempo que ajuda profissionais do ramo de trabalhos domésticos a encontrar um emprego. A plataforma suporta várias categorias de trabalhos domésticos, incluindo, mas não limitado a, empregados(as) domésticas, babás e cozinheiros(as).

%% Comente sobre a empresa: o setor de atuação, o porte, a missão e outras informações que você julgue necessárias, como por exemplo plano de carreira e premiações.

%% Em toda a monografia, você pode fazer uso de  referencias bibliográficas para apresentar URLs~\cite{WAI}, livros~\cite{Dahl:1972}~\cite{Hopcroft:1969} ou artigos utilizados no decorrer do estágio, publicados em periódicos~\cite{Almorsy:2012} ou revistas~\cite{MichettiJavaMagazine2013}.

\section{Sobre a Processo Seletivo}

Fui indicado à vaga pelo Gustavo Aguiar, um amigo que conheceu o meu atual supervisor, Rafael, num evento. O processo seletivo incluiu uma entrevista técnica, um desafio prático envolvendo área de interesse e outras duas entrevistas com os outros sócios da empresa.

Para me preparar para as entrevistas, eu busquei informações sobre a empresa e o que eles faziam, além de revisar alguns conceitos de infraestrutura e DevOps para responder as perguntas técnicas.

Para me o desafio, tive que reforçar meus conhecimentos em Docker e sistemas Linux.

O Rafael teve uma resposta positiva à entrevista e minha solução do desafio, me convidando para as entrevistas com os outros dois sócios e eventualmente desencadeando na minha contratação para a vaga de estágio.

%% Comente sobre como foi o processo seletivo para admissão no estágio. Discuta como você se preparou para ele.

\section{Apresentação da monografia}

%% Escreva um parágrafo que resume cada um dos demais capítulos da monografia, fazendo referência a cada um deles --- como no exemplo que segue. Indique também a existência de apêndices e anexos, se houver.

%% No Capítulo~\ref{chap:atividadesPlanejadas} é primeiramente sumarizado o planejamento das atividades previstas para todo o estágio, com o cronograma correspondente. A seguir, são apresentados  os treinamentos previstos.
