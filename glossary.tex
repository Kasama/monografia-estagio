% abbreviations: coloque em ordem alfabética
\newacronym{AWS}{AWS}{Amazon Web Services}
\newacronym{BCC}{BCC}{Bacharelado em Ciências da Computação}
\newacronym{CD}{CD}{Continuous Deployment}
\newacronym{CEP}{CEP}{Código de Endereçamento Postal}
\newacronym{CSS}{CSS}{Cascating Style Sheets}
\newacronym{DNS}{DNS}{Domain Name System}
\newacronym{EC2}{EC2}{Elastic Compute Cloud}
\newacronym{ECR}{ECR}{\gls{EC2} Container Registry}
\newacronym{ECS}{ECS}{\gls{EC2} Container Service}
\newacronym{HTML}{HTML}{Hypertext Markup Language}
\newacronym{HTTP}{HTTP}{Hypertext Transfer Protocol}
\newacronym{ICMC}{ICMC}{Instituto de Ciências Matemáticas e de Computação}
\newacronym{IO}{E/S}{Entrada e Saída}
\newacronym{LXC}{LXC}{Linux Containers}
\newacronym{POC}{PoC}{Proof of Concept}
\newacronym{S3}{S3}{Amazon Simple Storage Service}
\newacronym{SO}{SO}{Sistema Operacional}
\newacronym{SP}{SP}{São Paulo}
\newacronym{TDD}{TDD}{Test Driven Development}
\newacronym{TI}{TI}{Tecnologia da Informação}
\newacronym{USP}{USP}{Universidade de São Paulo}
\newacronym{scheduler}{Scheduler}{Serviço de Agendamento}

\glsaddall

\newglossaryentry{Docker}
{
  name=Docker,
  description={Docker é a empresa que está liderando o movimento do uso de \gslpl{container}. Docker permite a independencia entre aplicações e infraestrutura, facilitando a colaboração e inovação dos profissionais de desenvolvimento e de \gsl{ops}.},
  plural=Dockers
}

\newglossaryentry{container}
{
  name=container,
  plural=containers,
  description={\gls{LXC} é uma camada de virtualização no nível de \gls{SO} que permite a execução de múltiplos ambientes isolados no mesmo \gls{Kernel}
  }
}

\newglossaryentry{cluster swarm}
{
  name=cluster swarm,
  plural=clusters swarm,
  description={
    \gls{Docker} swarm é uma ferramenta que permite a execução de \glspl{container} em máquinas diferentes de forma orquestrada.
  }
}

\newglossaryentry{Kernel}
{
  name=Kernel,
  plural=Kernels,
  description={
    O Kernel é o programa de computador central de um \gls{SO}, que gerencia toda a comunicação entre os processos em alto nível e os dispositivos de \gls{IO}
  }
}

\newglossaryentry{ops}
{
  name=Ops,
  plural=Ops,
  description={
    Ops é o nome dado ao conjunto de processos e serviços que garantem o bom funcionamento da infraestrutura de um ambiente operacional em \gsl{TI}.
  }
}

\newglossaryentry{devops}
{
  name=DevOps,
  plural=DevOps,
  description={
    DevOps é o nome dado ao conjunto de atividades de desenvolvimento e \gls{ops}. O título profissional de DevOps se refere ao profissional de \gls{ops} que tem como tarefas implementar sistemas de entrega contínua, monitoramento de saúde de sistemas de produção e automação de processos operacionais, entre outras atividades.
  }
}

\newglossaryentry{Jenkins}
{
  name=Jenkins,
  plural=Jenkins,
  description={
    Jenkins é uma aplicação de código aberto que ajuda na automação de tarefas relacionadas a build, teste e distribuição de software.
  }
}

\newglossaryentry{OAuth2}
{
  name=OAuth2,
  plural=OAuth2,
  description={
    OAuth2 é um framework de autenticação que permite um controle de acesso granular entre o provedor de acesso e um cliente
  }
}

\newglossaryentry{framework}
{
  name=framework,
  plural=frameworks,
  description={
    Framework é uma abstração de software que provê uma funcionalidade genérica que pode ser incrementada por código do usuário.
  }
}

\newglossaryentry{backend}
{
  name=backend,
  plural=backends,
  description={
    Backend é a camada de um software responsável pelo acesso e estruturação de dados
  }
}

\newglossaryentry{frontend}
{
  name=frontend,
  plural=frontends,
  description={
    Frontend é a camada de um software responsável pela apresentação dos dados fornecidos pelo \gls{backend}
  }
}

\newglossaryentry{Teste AB}
{
  name=Teste AB,
  plural=Testes AB,
  description={
    Teste AB é um teste com o objetivo de medir métricas sobre uma mudança na plataforma, comparando-a com as métricas da plataforma antiga. O teste divide uma porcentagem dos usuários para cada hipótese e mede um objetivo para saber qual das hipóteses tem uma melhor performance para a audiência.
  }
}

\newglossaryentry{checkstyle}
{
  name=checkstyle,
  plural=checkstyles,
  description={
    Checkstyle é uma ferramenta de análise estática de código, que permite verificar regras de estilo de código para garantir consistência no código de um time
  }
}

\newglossaryentry{SCRUM}
{
  name=SCRUM,
  plural=SCRUM,
  description={
    SCRUM é uma metodologia ágil que tem como foco times de desenvolvimento de software entre três e nove membros que organizam suas tarefas em ciclos de duração fixa, chamados sprints.
  }
}

\newglossaryentry{javascript}
{
  name=javascript,
  plural=javascript,
  description={
    Javascript é uma linguagem de programação dinâmica focada no desenvolvimento web \gls{frontend}, apesar de ser utilizada em muitos outros cenários. Ela é desde 1996 a linguagem padrão nos navegadores web.
  }
}

\newglossaryentry{Vue}
{
  name=Vue JS,
  plural=Vue JS,
  description={
    Vue JS é um \gls{framework} para desenvolvimento \gls{frontend} baseado no conceito de componentes reutilizáveis.
  }
}

\newglossaryentry{Node}
{
  name=Node JS,
  plural=Node JS,
  description={
    Node JS é uma plataforma de execução de \gls{javascript} no \gls{backend}, baseada no motor de \gls{javascript} do navegador Google Chrome.
  }
}

\newglossaryentry{PHP}
{
  name=PHP,
  plural=PHP,
  description={
    PHP é uma linguagem de programação de propósito geral, focada no desenvolvimento web.
  }
}

\newglossaryentry{backup}
{
  name=backup,
  plural=backups,
  description={
   Backup é a ação de realizar cópia de dados de uma forma que eles possam ser restaurados no caso de uma perda, seja essa por falha de sistema, falha humana, acidentes naturais ou ataques maliciosos.
  }
}

\newglossaryentry{cookie}
{
  name=cookie,
  plural=cookies,
  description={
   Um cookie é uma quantidade pequena de dados enviados de um servidor para ser armazenado pelo usuário, os cookies geralmente são usados para identificar um usuário entre vários pedidos \gls{HTTP}, já que o protocolo não prevê armazenamento de estado entre requisições.
  }
}

\newglossaryentry{cronjob}
{
  name=cron job,
  plural=cron jobs,
  description={
    Cron é um programa muito utilizado em sistemas tipo \gls{UNIX} para execução de scripts de forma periódica em horários específicos com uma sintaxe flexível que possibilita regras complexas de intervalos.
  }
}

\newglossaryentry{UNIX}
{
  name=UNIX,
  plural=UNICES,
  description={
    UNIX é uma família de \glspl{SO} multitarefa e multiusuário. O termo é normalmente usado para definir sistemas operacionais tipo UNIX, que são aqueles que se assemelham e se comportam de forma muito próxima aos sistemas UNIX reais. Alguns exemplos de sistemas tipo UNIX incluem Linux, MacOS, BSD, Minix, entre outros.
  }
}
