\chapter{Conclusão}\label{chap:conclusao}

% Este capítulo não deve ultrapassar o total de 2 páginas.

\section{Benefícios para o crescimento profissional}

O trabalho realizado me permitiu um grande crescimento profissional, pois foi meu primeiro contato com um trabalho em bem organizado no qual minhas ações tinham um impacto real no time e na empresa. A experiência também me proporcionou uma excelente oportunidade de aprendizado na área de \gls{ops}, com oportunidades, como o uso de alguns serviços da \gls{AWS}, que não seriam possíveis fora da empresa.\\

Um ponto de aprendizado que não era esperado veio da convivência com o time de marketing e de negócios. Foi possível aprender algumas terminologias e motivações para as atividades que eram desenvolvidas pelo time de produto. Tal visão auxiliou na motivação do trabalho, pois possibilitou enxergar o impacto real das tarefas.

% Comente como o estágio foi importante para o seu crescimento profissional.
%
% Se houve problemas e/ou decepções, discuta-os nesta seção.


\section{Considerações sobre o curso de graduação}

O curso de graduação oferecido pelo \gls{ICMC} consegue dar uma base de conhecimento muito sólida, a partir da qual os alunos podem se especializar, entretanto a graduação não incentiva muito bem essa especialização. Apesar de os últimos semestres serem focados em matérias optativas que deveriam servir para essa especialização, a abordagem dos assuntos é geralmente muito rasa e muitas áreas de interesse não são representadas nas opções possíveis.


foi satisfatório em termos de preparação para o estágio realizado, entretanto na maioria das disciplinas o aprendizado da matéria ocorreu apenas fora da sala de aula, pois as aulas apenas mostravam de forma superficial o conteúdo.\\

A graduação teve um foco muito grande na presença em sala de aula com trabalhos massantes e desmotivantes, muitas vezes sem aplicação prática. Tal foco acaba dificultando a possibilidade dos alunos de perseguirem projetos pessoais de extensão, um dos pilares da universidade.\\

% Avalie o seu curso de graduação, em particular em termos de preparação para o mercado de trabalho.
%
% Discuta também a importância das disciplinas do curso para o estágio realizado.

\section{Sugestões para o curso de graduação}

% Apresente sugestões construtivas para o curso de graduação, justificando cada uma.

O curso de graduação poderia ser estruturado de forma a incentivar o aprendizado mais aprofundado em sala de aula com um

\section{Planos para o futuro}

% Comente quais são seus planos depois do término do estágio, a curto e médio prazos.

Após o término do estágio, os planos a curto prazo são de trabalhar na área de \gls{ops}
