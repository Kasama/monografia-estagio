\chapter*{Resumo}

%TODO
\begin{doublespace}
\noindent{%Obs. 1: Item obrigatório;\\

O estágio na área de \gls{devops} foi planejado para auxiliar a \nomeEmpresa, uma agência de empregos domésticos online, a migrar de uma aplicação legada monolítica para um conjunto de micro serviços escaláveis e mais simples. Várias atividades foram realizadas durante o período de estágio para possibilitar essa migração, mas também foram executadas outras tarefas, como automação de processos internos e execução automática de testes. Também foi desenvolvido uma aplicação de agendamento de tarefas, que se provou muito útil e foi bastante utilizada para execução de \glspl{backup} e geração de relatórios. Foi possível aprender muito sobre infraestrutura de sistemas profissionais, o funcionamento de plataformas reais e como mantê-las acessíveis para um número alto de acessos concorrentes, usando a plataforma da \gls{AWS}, tal aprendizado seria praticamente impossível sem o auxílio da empresa. Para a o planejamento e acompanhamento do trabalho foi utilizada a metodologia \gls{SCRUM} num ambiente amigável e descontraído. Ao final do período foi possível observar que muitos resultados interessantes foram atingidos, como a grande diminuição de avisos de \gls{checkstyle} e uma prévia dos resultados do \gls{Teste AB} entre a plataforma antiga e a nova.

% Obs. 2: O resumo deve apresentar de forma concisa os pontos relevantes do texto. Deve descrever, de maneira objetiva e sucinta, o objetivo do estágio, sua relevância para a formação do aluno e para empresa onde o estágio foi realizado, a metodologia empregada, e os resultados obtidos;\\
% Obs. 3: Deve ser redigido em parágrafo único, através de uma sequência de frases concisas e objetivas e com espaçamento duplo;\\
% Obs. 4: O resumo não deve ultrapassar 15 linhas ou 500 palavras (o menor);\\
% Obs. 5: Mais informações sobre a escrita de resumo, consulte a norma NBR 6028/1990 para escrita de resumos.\\
% Obs. 6: Note que, para cada abreviatura que você pretente utilizar em qualquer parte da monografia, a primeira vez que ela é utilizada deve ser sempre feita por extenso, como em \gls{USP}, \gls{ICMC} e \gls{BCC}.\\
% Obs. 7: depois é de boa \gls{USP}.\\
% Obs. 8: imagina o caso do \gls{EC2}. e \gls{ECS}
}
\end{doublespace}
%\end{resumo}

\noindent \textbf{Palavras-chaves:} \gls{devops}, Infraestrutura, Agendamento, Migração, \gls{Teste AB}
