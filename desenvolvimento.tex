\chapter{Desenvolvimento do Trabalho}\label{chap:atividadesRealizadas}

% Este capítulo não deve ultrapassar o total de 4 páginas.

\section{Atividades Realizadas}

Nas primeiras semanas de trabalho foi dada uma tarefa de implementação de um \gls{scheduler}, com uma interface para agendamento de scripts usando a sintaxe de \gls{cronjob}, como mostrada na Figura~\ref{fig:scheduler}. O \gls{scheduler} foi criado para funcionar dentro de um \gls{container} \gls{Docker}, no qual é possível controlar o estado de outros \glspl{container} sendo executados no mesmo sistema (ou \gls{cluster swarm}).\\

Para o uso na empresa, o \gls{scheduler} foi colocado em uma máquina virtual da \gls{EC2}, a partir da qual é possível uma comunicação segura com outras instâncias na \gls{EC2}, permitindo a execução de serviços sensíveis sem a necessidade de abertura dos canais de comunicação do servidor de produção para o mundo externo.\\

Esse \gls{scheduler} foi amplamente utilizado em tarefas futuras, para automação de processos recorrentes.\\

O \gls{scheduler} tem código aberto e está disponível em \url{https://github.com/casa-e-cafe/docker-scheduler}\\

\begin{figure}[h]
  \rule[1ex]{\textwidth}{0.25pt}
  \centering\includegraphics[width=1.00\textwidth]{img/scheduler.png}
  \caption[Scheduler]
  {Scheduler}\label{fig:scheduler}
  \rule[1ex]{\textwidth}{0.25pt}
\end{figure}

Enquanto essa tarefa era desenvolvida, foram explicadas partes da arquitetura e organização da plataforma nas reuniões diárias do \gls{SCRUM}. Também ocorriam momentos nos quais eram ensinados procedimentos de uso do \gls{AWS}, principalmente do \gls{EC2} e \gls{S3}, esses momentos foram de suma importância para a realização de tarefas futuras, pois a \nomeEmpresa~utiliza serviços do \gls{AWS} para a maior parte da aplicação.\\

A segunda grande tarefa realizada foi a implementação do \gls{OAuth2} no \gls{backend} e \gls{frontend}. Em ambos os casos o desenvolvimento foi realizado em conjunto com outro membro mais experiente do time, de forma pareada, afim de promover a familiarização com o código e arquitetura existente.\\

Após esse período de familiarização, começaram a ser atribuídas tarefas da área de \gls{ops}. A primeira se resumiu a usar o \gls{scheduler} para automatizar a geração de relatórios de uso da plataforma, que antes eram gerados de forma manual semanalmente. Foi para o armazenamento dos relatórios, foi utilizado o \gls{S3}.\\

Outra importante tarefa na área de \gls{ops} foi a implementação do \gls{Teste AB}. Tal teste tem como objetivo medir a taxa de conversão da versão antiga da platforma comparada com a versão nova. A implementação ocorreu em duas etapas. Primeiramente houve uma \gls{POC} para entender ferramentas que poderiam ser usadas. Depois ocorreu a implementação do código em si. Foi decidido que o \gls{Teste AB} seria implementado diretamente em \gls{PHP} no código da aplicação antiga, permitindo controle de separação por porcentagem de acessos e múltiplas variações para o teste.\\

O código na Listagem~\ref{src:abtest} ilustra a parte principal da implementação, que verifica se a requisição \gls{HTTP} possui um \gls{cookie} de identificação do teste, ou se é necessário gerar um novo baseado nas variações e probabilidades configuradas\\

\begin{lstlisting}[
  language=PHP,
  caption=Teste AB,
  label=src:abtest
]
if ($existing_cookie) {
  return $variations[$existing_cookie]['url'];
} else {
  $random_seed = $this->get_random_seed($settings['seed_method']);
  $variation = $this->select_variation($random_seed, $variations);

  $new_cookie = array(
      'name' => $settings['cookie']['name'],
      'value' => $variation,
      'expire' => $settings['cookie']['expiration'],
      'path' => '/'
      );
  $this->input->set_cookie($new_cookie);
  return $variations[$variation]['url'];
}
\end{lstlisting}

Após a implementação do \gls{Teste AB}, foi iniciada a implantação do sistema de build e testes \gls{Jenkins} para que houvesse a automatização dos testes programáticos e \gls{checkstyle}, assim como geração de imagens \gls{Docker} do projeto. Com objetivo de atingir \gls{CD}\\

O planejamento inicial incluiu apenas a execução de testes e verificação de \gls{checkstyle}, para todas as tecnologias usadas no produto: \gls{PHP}, \gls{Node} e \gls{Vue}. A Figura~\ref{fig:jenkinsBuild} mostra o exemplo de uma build realizada com o \gls{Jenkins}\\

\begin{figure}[h]
  \rule[1ex]{\textwidth}{0.25pt}
  \centering\includegraphics[width=0.70\textwidth]{img/JenkinsBuild.png}
  \caption[Jenkins Build]
  {Jenkins Build}\label{fig:jenkinsBuild}
  \rule[1ex]{\textwidth}{0.25pt}
\end{figure}

Foram realizadas duas tarefas grandes de \gls{backup}, uma do blog da empresa e outra dos dados da base de produção. Em ambos os casos foi escolhido usa o \gls{S3} para armazenamento dos dados copiados, principalmente pelo baixo custo e facilidade de uso.\\


% Descreva quais atividades foram realizadas durante o projeto.
%
% Para cada atividade realizada, discuta também qual foi a dinâmica de trabalho --- por exemplo, se foi utilizada a metodologia SCRUM ou algo similar.
%
% Se apropriado, você apresentar algoritmos ou códigos, como o ilustrado na Listagem~\ref{src:code1}, mas deve explicar o seu funcionamento no texto.

% \begin{lstlisting}[
%   language=C,
%   caption=Exemplo de código,
%   label=src:code1
% ]
% #include <stdio.h>
% int main(){
%   int i=0, j=1;
%   printf("i:%d j:%d\n",i,j);
%   return;
% }
% \end{lstlisting}
%
% Você também pode fazer uso de figuras (como a Figura~\ref{fig:fig1}), mas deve explicar a figura no texto.
%
% \begin{figure}[!ht]
%   \rule[1ex]{\textwidth}{0.25pt}
%   \centering\includegraphics[width=0.25\textwidth]{img/logoICMC.png}
%   \caption[Exemplo de figura]
%   {Exemplo de figura}\label{fig:fig1}
%   \rule[1ex]{\textwidth}{0.25pt}
% \end{figure}


\section{Problemas resolvidos}

Os principais problemas que foram resolvidos na duração do estágio foram principalmente relacionados a automatização de procedimentos.\\

A implementação do \gls{scheduler} facilitou a realização de muitas tarefas, permitindo o agendamento de atividades recorrentes com facilidade. Com ele foi possível solucionar o problema de geração manual de relatórios toda a semana, permitindo que não fosse necessário executar vários scripts manualmente. Também foi possível, com o \gls{scheduler}, realizar \glspl{backup} diários sem o custo de alocar uma pessoa para essa tarefa.\\

Outra solução importante gerada a partir do trabalho realizado foi a possibilidade de verificar resultados de testes automatizados e \gls{checkstyle} na interface do \gls{Jenkins}. Na Figura~\ref{fig:jenkinsCheckstyle} é possível ver um dos gráficos gerados pelo sistema. No eixo horizontal podemos ver o número da build e no vertical a quantidade de avisos de alta prioridade (em vermelho) e baixa prioridade (em amarelo) de \gls{checkstyle}. É possível notar que houve uma diminuição grande na quantidade de avisos com o tempo.\\

\begin{figure}[h]
  \rule[1ex]{\textwidth}{0.25pt}
  \centering\includegraphics[width=0.70\textwidth]{img/Jenkins.png}
  \caption[Jenkins Checkstyle Trend]
  {Jenkins Checktyle Trend}\label{fig:jenkinsCheckstyle}
  \rule[1ex]{\textwidth}{0.25pt}
\end{figure}

\section{Técnicas, métodos e tecnologias envolvidas}

Durante todo o período do estágio, foi utilizada metodologia de desenvolvimento \gls{SCRUM}. Os primeiros sprints tiveram duração de uma semana em um time com 5 desenvolvedores e um \gls{SCRUM} Master, que ocasionalmente realizava tarefas do sprint. Após um tempo o time decidiu adotar sprints de duas semanas, mas foi retomada a duração inicial após três sprints. Para a organização do time, são utilizadas ferramentas como \href{https://trello.com/}{Trello} para organização de tarefas, \href{https://gitlab.com}{GitLab} para organização de código fonte, \href{https://slack.com/}{Slack} para comunicação dentro do time e \href{https://www.nuclino.com/}{Nuclino} para documentação.\\

Para a infraestrutura, são usados vários serviços da \href{https://aws.amazon.com/}{\gls{AWS}}, como \href{https://aws.amazon.com/ec2/}{\gls{EC2}} para hospedagem dos serviços, \href{https://aws.amazon.com/ecr/}{\gls{ECR}} para hospedagem de imagens \gls{Docker}, \href{https://aws.amazon.com/cloudwatch/}{Cloud Watch} para monitoramento de eventos em produção e \href{https://aws.amazon.com/s3/}{\gls{S3}} para armazenamento de relatórios e backups. Também são usados vários outros serviços não ligados a Amazon, como \href{https://www.hostgator.com.br/}{Hostgator} para hospedagem da plataforma legada e \href{https://www.cloudflare.com/br/}{Cloud Flare} para gerenciamento de \gls{DNS}.\\

As tecnologias envolvidas na plataforma podem ser separadas em três categorias. A primeira são as tecnologias de \gls{frontend}, que incluem \gls{HTML}, \gls{CSS} e \gls{javascript} com \gls{Vue}. No \gls{backend} são usadas as linguagens de programação \gls{PHP} com \href{https://codeigniter.com/}{CodeIgniter} e \href{https://silex.symfony.com/}{Silex} e \gls{javascript} com \gls{Node}. Na terceira categoria se encaixam tecnologias de apoio e infraestrutura, como \href{https://www.docker.com/}{\gls{Docker}} para a execução dos processos em \glspl{container}, \href{https://www.mongodb.com/}{Mongo DB} e \href{https://www.mysql.com/}{MySql} para as bases de dados, \href{https://docs.docker.com/docker-for-aws/}{\gls{Docker} for \gls{AWS}} para a integração com as máquinas da \gls{EC2}, \href{https://traefik.io/}{Træfik} para gerenciamento de trafego nos ambientes de teste e \href{https://analytics.google.com/analytics/}{Google Analytics} para medir métricas da plataforma\\

Para o desenvolvimento de novas funcionalidades, é utilizado o método \gls{TDD}, que visa a criação de testes da aplicação antes de iniciar o desenvolvimento de uma nova funcionalidade. Com testes unitários é possível facilitar o desenvolvimento, por diminuir o tempo de teste.\\

% Descreva os métodos, técnicas  e tecnologias
% envolvidos  ou que  foram utilizados  para a  condução das  atividades
% durante  o estágio.  Por exemplo:  (i)  no caso  dos métodos,  pode-se
% apresentar XP (eXtreme Programming) e SCRUM, métodos empregados para o
% teste de  sistemas, entre  outros; (ii) No  caso de  técnicas, pode-se
% descrever técnicas da UML, técnicas de teste, entre outros; e (iii) no
% caso  de  tecnologias,  pode-se descrever  ferramentas  (por  exemplo,
% Spring, Hibernate,  Struts, entre  outros), linguagens  de programação
% (por exemplo, Java), padrões de projeto utilizadas, entre outros.
%
% Faça  referências bibliográficas  atuais e  de fontes
% relevantes  (evite  sites,  privilegie   livros  ou  artigos);  mostre
% diferentes  tecnologias e  faça comparações,  quando for  o caso.  Não
% esquecer de citar a fonte corretamente, por exemplo~\cite{MichettiJavaMagazine2013}.

\section{Impacto}

Os principais afetados diretamente pelo trabalho realizado foram os desenvolvedores e a área administrativa da empresa, que tiveram muitas tarefas diárias facilitadas e automatizadas, permitindo que pudessem utilizar seu tempo de forma mais eficiente para produzir um melhor conteúdo para os clientes.\\

Os clientes também foram afetados diretamente em algumas situações, como no caso do \gls{Teste AB}, Que disponibilizou o acesso a nova plataforma para medir seu impacto. Apesar de ainda não haver significância estatística nos dados, a Figura~\ref{fig:analytics} mostra alguns dados coletados pelo Google Analytics e a comparação entre a plataforma antiga (Monolito, em azul) e a nova (Vue, em laranja). É possível ver uma diminuição na taxa de rejeição e aumento no engajamento, mas é necessário reforçar que os dados não tem significância estatística, por causa do baixo número de visitas redirecionadas para a plataforma nova.\\

\begin{figure}[h]
  \rule[1ex]{\textwidth}{0.25pt}
  \centering\includegraphics[width=0.80\textwidth]{img/analytics.png}
  \caption[Dados do Teste AB]
  {Dados do Teste AB}\label{fig:analytics}
  \rule[1ex]{\textwidth}{0.25pt}
\end{figure}

% Quais foram as pessoas/enfoitidades afetadas por esses resultados? Quem são  as pessoas/entidades que potencialmente serão afetadas?
%
% Fale sobre a relevância da solução do(s) problema(s) para a empresa e seu(s) cliente(s).
%
% Você deve omitir informações sigilosas.
%
% Discuta também sobre a relevância para sua formação a  participação na solução do(s) problema(s).

\section{Problemas não resolvidos}

Das tarefas planejadas para o estágio, poucas não puderam ser desenvolvidas. Uma delas diz respeito ao sistema de build \gls{Jenkins}. Não foi possível adicionar informações de cada build no GitLab, o que facilitaria na visualização das falhas de testes ou \gls{checkstyle}, sem a necessidade de entrar na interface do \gls{Jenkins}.\\

Algumas atividades que estavam originalmente no cronograma, como a alteração da arquitetura para utilizar um padrão orientado a eventos, não foram possíveis de serem realizadas por conta de outras tarefas terem uma prioridade mais elevada.\\
